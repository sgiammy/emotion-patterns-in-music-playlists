% CHAPTER 1 - INTRODUCTION
\chapter{Introduction}
\section{Background}
In the last few years the online music streaming services such as Spotify, Apple Music and Deezer introduced, among others,  the possibility to create playlists thus opening new challenges on music recommendation.\par
One of the new possible tasks a modern Recommender System should perform is automatic playlist continuation. By suggesting appropriate songs to add to a playlist, a Recommender System can increase user engagement by making playlist creation easier, as well as extending listening beyond the end of existing playlists. \par
More precisely the task of automatic playlist continuation consists in: given a set of playlist features, the system shall generate a list of recommended tracks that can be added to that playlist, thereby ``continuing'' the playlist. \par

\section{Project scope}
One of the possible features to consider in developing a system for the automatic playlist continuation task is the emotion expressed in each song of the playlist and the more frequent transition patterns from one emotion to the other. Thus, not only the emotion of each song lyrics must be detected, but also the transitions between emotions must be analyzed. \par
Emotion Detection is a novel and promising field of study of Natural Language Understanding, which is able to automatically infer what are the emotions expressed in a text. It can be considered as a Sentiment Analysis task, which is the computational treatment of opinions, sentiments and subjectivity of a natural language text. \par

\section{Results}

Below we just reported the accuracy results we obtained throughout our project while classifying emotions both in song lyrics and in playlists, as shown in table \ref{tab:compar} and \ref{tab:compar2} respectively.

\begin{table}[H]
\centering
\begin{tabular}{ |p{3cm}||p{1.5cm}|p{1.5cm}|p{1.5cm}|p{1.5cm}|  }
 \hline
 \multicolumn{5}{|c|}{10-fold Cross Validation Accuracy} \\
 \hline
 Dataset & ANN & LR &SVM & xgboost\\
 \hline
MoodyLyrics  & 90.55\%    &- &  90\% & 86\%\\
MoodyLyrics4Q  & 55.97\%    &57.87\% &  58.04\% & 56.89\%\\
Both together &   68.41\%  & 69.42\%   &69.32\% &64.27\%\\
\hline
\end{tabular}
\caption{Emotion detection accuracies} \label{tab:compar}
\end{table}

\begin{table}[H]
\centering
\begin{tabular}{ |p{3cm}||p{1.5cm}|p{1.5cm}| }
 \hline
 \multicolumn{3}{|c|}{Playlist Classification Accuracy} \\
 \hline
Dataset & With outliers & Without outliers\\
 \hline
MoodyLyrics & 29\% & 29\%\\
MoodyLyrics4Q  & 66\%    &66\%\\
Both together &   50\%  & 47\%\\
\hline
\end{tabular}
\caption{Playlist classification accuracies} \label{tab:compar2}
\end{table}

We will go into the details of how those results were obtained in the next chapters of this report.

\section{Report outline}
The report is structured as follow: \textit{chapter 2} contains the analysis of the state of the art in emotion detection tasks. \textit{Chapter 3} includes an exploration of the background knowledge for this project, \textit{chapter 4} describes the lyrics emotion classification approaches tried, \textit{chapter 5} the playlists classification methods and results while \textit{chapter 6} contains conclusions and future works considerations. 










