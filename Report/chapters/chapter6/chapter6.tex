\chapter{Conclusions and Future Works}

In this report we detailed all the experiments and trials we did in the context
of our semester project. Our main goal was to firstly build a system capable of
classifying songs lyrics based on their emotion, and then to use the acquired knowledge 
to understand emotional patterns in music playlists.

The first thing we did, was focusing on the word embedding vectors generated for the lyrics
of the already labeled songs we found in the MoodyLyrics dataset, which classifies lyrics according to 
4 possible emotions: angry, sad, relaxed and happy. As we understood that this
approach was very limiting, we started working on some smarter feature engineering approaches,
in which we extracted our own features from the analyzed lyrics.

Under the suggestion of the dataset's creator, we then moved from MoodyLyrics to its updated and more
reliable version: MoodyLyrics4Q. This new dataset became then the gold standard for our experiments.

We tried to combine our lyrics dataset also with data coming from different data sources, e.g. tweets. However
we understood that the best way to proceed in our project was to keep working with just the lyrics
based dataset, as it is obviously the most suitable one for unraveling emotional patters in music playlists.

At the end of the lyrics classification process, our output was, for each song, an emotion vector, composed 
of the intensities at which each sentiment was expressed.

Once we understood which was the best subset of features and the best classifiers, we moved our focus on
classifying playlists. For evaluating our playlist classification performances, we compared our outputs
to those of the silver standard playlist dataset we built.

We used two different approaches for playlist classification: in the first one we
just computed the emotion classification vector for each song in the playlist and then we averaged them;
in the second approach, before doing the average, we exclude from the emotion vectors those values which 
were considered to be too different from the others (outliers).

Before having a look at the emotional patterns in the playlists, we tried to ease out our computational
load by trying to see if it was possible to properly classify a playlist based on just a subset of its songs.
However, we found out that we need all the songs in each playlist if we want to achieve the best results possible.

To conclude, we explored the emotional patterns inside each playlist, meaning that we saw how the emotion
classification vector evolved inside each playlist, transitioning from one emotion to another.

In conclusion, the results we obtained are shown in Table \ref{tab:compar6} and Table \ref{tab:compar62}.

\begin{table}[H]
\centering
\begin{tabular}{ |p{3cm}||p{1.5cm}|p{1.5cm}|p{1.5cm}|p{1.5cm}|  }
 \hline
 \multicolumn{5}{|c|}{10-fold Cross Validation Accuracy} \\
 \hline
 Dataset & ANN & LR &SVM & xgboost\\
 \hline
MoodyLyrics  & 90.55\%    &- &  90\% & 86\%\\
MoodyLyrics4Q  & 55.97\%    &57.87\% &  58.04\% & 56.89\%\\
Both together &   68.41\%  & 69.42\%   &69.32\% &64.27\%\\
\hline
\end{tabular}
\caption{Emotion detection accuracies} \label{tab:compar6}
\end{table}

\begin{table}[H]
\centering
\begin{tabular}{ |p{3cm}||p{1.5cm}|p{1.5cm}| }
 \hline
 \multicolumn{3}{|c|}{Playlist Classification Accuracy} \\
 \hline
Dataset & With outliers & Without outliers\\
 \hline
MoodyLyrics & 29\% & 29\%\\
MoodyLyrics4Q  & 66\%    &66\%\\
Both together &   50\%  & 47\%\\
\hline
\end{tabular}
\caption{Playlist classification accuracies} \label{tab:compar62}
\end{table}


%%%%%%%%%%%%%%%%%%%%%%%%%%%%%%%%%%%%%%%%%%%%%%%%%%%%
%%%%%%%%%%%%%%%%%%%%%%%%%%%%%%%%%%%%%%%%%%%%%%%%%%%%
%%%%%%%%%%%%%%%%%%%%%%%%%%%%%%%%%%%%%%%%%%%%%%%%%%%%

Despite the interesting results we obtained, there is still much room
for improvements.

Indeed, one thing we did not explore in much depth is the usage of Recurrent 
Neural Networks (RNN) and Convolutional Neural Networks (CNN), employed by
many modern systems built for emotion classification tasks.

We made several small experiments using some complex CNN and RNN based networks
we found in the literature\cite{text-emotion-classification}. However, those models produced very poor performances,
which is the reason why we did not even mention them in the current report.

Anyway, those small experiments served to the purpose of letting us understand 
which direction to take in the future. Indeed, neural networks components, such as
Long Short Term Memories (LSTMs), seems to be perfectly suited for those kind of problems
and, if properly tuned, may help us in reaching better performances.

One current limitation we saw when running the mentioned experiments with deep neural
networks is the shortage of training data. Indeed, effective training of 
neural networks requires a huge amount of data. In the low-data regime, parameters 
are underdetermined, and learnt networks generalise poorly. 

The simplest way to solve this problem, would be to manually label more and more song lyrics, eventually using
automated mechanisms e.g. Amazon Mechanical Turk\cite{amazon-turk}. However this solution is not practical
and expensive (especially if we use Amazon Mechanical Turk).

Certainly, a better approach would be to use some automated data generation techniques.
Indeed, we may think of using some of the Data Augmentation mechanisms currently employed in deep learning. 
We found several interesting techniques\cite{DBLP:journals/corr/abs-1711-00648} which could be employed in our domain
and could help alleviate issue by using Generative Adversarial Neural Networks (GANN), which
are able to generate new data from the already existing one.
