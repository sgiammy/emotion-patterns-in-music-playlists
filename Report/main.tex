%%%%%%%%%%%%%%%%%%%%%%%%%%%%%%%%%%%%%%%%%
% University Assignment Title Page 
% LaTeX Template
% Version 1.0 (27/12/12)
%
% This template has been downloaded from:
% http://www.LaTeXTemplates.com
%
% Original author:
% WikiBooks (http://en.wikibooks.org/wiki/LaTeX/Title_Creation)
%
% License:
% CC BY-NC-SA 3.0 (http://creativecommons.org/licenses/by-nc-sa/3.0/)
% 
% Instructions for using this template:
% This title page is capable of being compiled as is. This is not useful for 
% including it in another document. To do this, you have two options: 
%
% 1) Copy/paste everything between \begin{document} and \end{document} 
% starting at \begin{titlepage} and paste this into another LaTeX file where you 
% want your title page.
% OR
% 2) Remove everything outside the \begin{titlepage} and \end{titlepage} and 
% move this file to the same directory as the LaTeX file you wish to add it to. 
% Then add \input{./title_page_1.tex} to your LaTeX file where you want your
% title page.
%
%%%%%%%%%%%%%%%%%%%%%%%%%%%%%%%%%%%%%%%%%
%\title{Title page with logo}
%----------------------------------------------------------------------------------------
%   PACKAGES AND OTHER DOCUMENT CONFIGURATIONS
%----------------------------------------------------------------------------------------

\documentclass[12pt]{report}
\usepackage[english]{babel}
\usepackage[utf8x]{inputenc}
\usepackage{amsmath}
\usepackage{graphicx}
\usepackage[colorinlistoftodos]{todonotes}
\usepackage{float}
\usepackage{hyperref}
\usepackage{listings}
\usepackage{booktabs}
\usepackage{subcaption}
\usepackage{cite}

\begin{document}

\begin{titlepage}

\newcommand{\HRule}{\rule{\linewidth}{0.2mm}} % Defines a new command for the horizontal lines, change thickness here

\center % Center everything on the page
 
%----------------------------------------------------------------------------------------
%   HEADING SECTIONS
%----------------------------------------------------------------------------------------

%\textsc{\LARGE EURECOM}\\[1.5cm] % Name of your university/college
\textsc{\Large Semester Project Report}\\[0.5cm] % Major heading such as course name
\textsc{\large Department Of Data Science and Engineering}\\[0.5cm] % Minor heading such as course title

%----------------------------------------------------------------------------------------
%   TITLE SECTION
%----------------------------------------------------------------------------------------

\HRule \\[0.4cm]
{ \Large \bfseries Emotion Patterns in Music Playlists}\\[0.4cm] % Title of your document
\HRule \\[6cm]
 
%----------------------------------------------------------------------------------------
%   AUTHOR SECTION
%----------------------------------------------------------------------------------------

\begin{minipage}{0.4\textwidth}
\begin{flushleft} \large
\emph{Authors:}\\
Sara \textsc{Giammusso} \\ Mario \textsc{Guerriero} % Your name
\end{flushleft}
\end{minipage}
~
\begin{minipage}{0.4\textwidth}
\begin{flushright} \large
\emph{Supervisors:} \\
Raphael \textsc{Troncy}\\ Pasquale \textsc{Lisena}\\ Enrico \textsc{Palumbo} % Supervisor's Name
\end{flushright}
\end{minipage}\\[2cm]

% If you don't want a supervisor, uncomment the two lines below and remove the section above
%\Large \emph{Author:}\\
%John \textsc{Smith}\\[3cm] % Your name

%----------------------------------------------------------------------------------------
%   DATE SECTION
%----------------------------------------------------------------------------------------

{\large \today}\\[2cm] % Date, change the \today to a set date if you want to be precise

%----------------------------------------------------------------------------------------
%   LOGO SECTION
%----------------------------------------------------------------------------------------

\includegraphics[width=0.3\textwidth]{logo.png}\\[1cm] % Include a department/university logo - this will require the graphicx package
 
%----------------------------------------------------------------------------------------

\vfill % Fill the rest of the page with whitespace

\end{titlepage}

\tableofcontents


\begin{abstract}

Music streaming services such as Spotify are revolutionizing the music world, enabling a transition from artist-created bundles of songs (CDs) to user-created playlists.\par
Different logics may be applied in the generation of a playlist: they can contain songs of a similar genre (e.g. ``Rock playlist''), fit to a particular occasion (e.g. ``New year's eve party''), to a particular context (e.g. ``Gym''), to a particular mood (e.g.``Happy'') and so on.\par

The goal of this semester project is to unravel the emotion patterns underlying the sequences of songs in a playlist using automatic approaches of Emotion Detection on the lyrics.

\end{abstract}
\newpage

% CHAPTER 1 - INTRODUCTION
\chapter{Introduction}
\section{Background}
In the last few years the online music streaming services such as Spotify, Apple Music and Deezer introduced, among others,  the possibility to create playlists thus opening new challenges on music recommendation.\par
One of the new possible tasks a modern Recommender System should perform is automatic playlist continuation. By suggesting appropriate songs to add to a playlist, a Recommender System can increase user engagement by making playlist creation easier, as well as extending listening beyond the end of existing playlists. \par
More precisely the task of automatic playlist continuation consists in: given a set of playlist features, the system shall generate a list of recommended tracks that can be added to that playlist, thereby ``continuing'' the playlist. \par

\section{Project scope}
In the light of this, one of the possible features to consider in developing a system for the automatic playlist continuation task is the emotion expressed in each song of the playlist and the more frequent transition patterns from one emotion to the other. Thus, not only the emotion of each song lyrics must be detected, but also the transitions between emotions must be analyzed. \par
Emotion Detection is a novel and promising field of study of Natural Language Understanding, which is able to automatically infer what are the emotions expressed in a text. It can be considered as a Sentiment Analysis task, which is the computational treatment of opinions, sentiments and subjectivity of a natural language text. \par

\section{Results}

Below we just reported the accuracy results we obtained throughout our project while classifying both lyrics and playlists, as shown in table \ref{tab:compar} and \ref{tab:compar2} respectivelly.

\begin{table}[H]
\centering
\begin{tabular}{ |p{3cm}||p{1.5cm}|p{1.5cm}|p{1.5cm}|p{1.5cm}|  }
 \hline
 \multicolumn{5}{|c|}{10-fold Cross Validation Accuracy} \\
 \hline
 Dataset & ANN & LR &SVM & xgboost\\
 \hline
MoodyLyrics  & 90.55\%    &- &  90\% & 86\%\\
MoodyLyrics4Q  & 55.97\%    &57.87\% &  58.04\% & 56.89\%\\
Both together &   68.41\%  & 69.42\%   &69.32\% &64.27\%\\
\hline
\end{tabular}
\caption{Emotion detection accuracies} \label{tab:compar}
\end{table}

\begin{table}[H]
\centering
\begin{tabular}{ |p{3cm}||p{1.5cm}|p{1.5cm}| }
 \hline
 \multicolumn{3}{|c|}{Playlist Classification Accuracy} \\
 \hline
Dataset & With outliers & Without outliers\\
 \hline
MoodyLyrics & 24\% & 24\%\\
MoodyLyrics4Q  & 71\%    &66\%\\
Both together &   58\%  & 61\%\\
\hline
\end{tabular}
\caption{Playlist classification accuracies} \label{tab:compar2}
\end{table}

We will go into the details of how those results were obtained in the next chapters of this report.

\section{Report outline}
The report is structured as follow: 
\begin{itemize}
\item \textit{Chapter 2}: analysis of the state of the art in emotion detection tasks
\item \textit{Chapter 3}: exploration of the background knoledge for this project
\item \textit{Chapter 4}: description of the lyrics emotion classification approaches tried
\item \textit{Chapter 5}: analysis of the state of the playlists classification results
\item \textit{Chapter 6}: conclusions and future works
\end{itemize}












\chapter{The State of the Art}

In the following chapter we will briefly go through the knowledge we acquired while exploring already existing materials in the same context of our problem.

\section{Sentiment Analysis} 
Sentiment Analysis (SA) is the computational study of people's opinions, attitudes and emotions toward an entity, the entity being an individual, event or topic.\cite{survey} \par

Sentiment Analysis can be considered a classification task as illustrated in Fig \ref{fig:sa_process}.\\

\begin{figure}[H]
\centering
\includegraphics[width=0.4\textwidth]{./chapters/chapter1/images/sa_process}
\caption{Sentiment analysis process on product reviews\cite{survey}}
\label{fig:sa_process}
\end{figure}

There are three main classification levels in SA: 
\begin{itemize}
\item Document-level
\item Sentence-level
\item Aspect-level
\end{itemize}
Their difference is the granularity at which they operate. Indeed, while document-level SA aims to classify a document as expressing a positive or negative opinion or sentiment by considering the whole document as the basic information unit, sentence-level SA aims at classifying the sentiments/opinions expressed in each sentence.
In both cases the first step is to identify whether the sentence/document is subjective or objective and if it is subjective determine whether the sentence expresses positive or negative opinions. \par

In certain applications, classifying text at the document level or at the sentence level may not provide the necessary details needed for detecting opinions on all aspects of the entity. Aspect-level SA, instead, aims at classifying the sentiment with respect to the specific aspects of entities. The first step is to identify the entities and their aspects. Then, all the different opinions on the same entity must be considered. Indeed, the opinion holders may also give different opinions for different aspects of the same entity, e.g. ``This chair is ugly but it is comfortable''.

Sentiment Analysis task is considered as a sentiment classification (SC) problem. The first step in the SC problem is to extract and select text features. Some of the most commonly used features are:
\begin{itemize}
\item \textbf{Terms presence and frequency}: These features are individual words or word n-grams and their frequency counts;
\item \textbf{Parts of speech (POS)}: finding adjectives, pronouns, etc. as they are important indicators of opinions;
\item \textbf{Opinion words and phrases}: these are words commonly used to express opinions including \textit{good or bad, like or hate}. On the other hand, some phrases express opinions without using opinion words, e.g. \textit{cost me and arm and a leg};
\item \textbf{Negations}: the appearance of negative words may change the opinion orientation like \textit{not good} is equivalent to \textit{bad}. 
\end{itemize}

Sentiment Classification techniques can be roughly divided into machine learning approach, lexicon based approach and hybrid approach\cite{survey}. \par

The \textit{Machine Learning Approach (ML)} applies the famous ML algorithms and uses linguistic features. 

The \textit{Lexicon-based Approach} relies on a sentiment lexicon, a collection of known and precomiled sentiment terms. It is divided into dictionary-based approach and corpus-based approach which use statistical or semantic methods to find sentiment polarity.

The \textit{Hybrid Approach} combines both approaches. 

The various approaches and the most popular algorithms of SC are illustrated in Fig \ref{fig:sentiment_classification}. 

\begin{figure}[H]
\centering
\includegraphics[width=0.8\textwidth]{./chapters/chapter1/images/sentiment_classification}
\caption{Sentiment classification techniques\cite{survey}}
\label{fig:sentiment_classification}
\end{figure}

The text classification methods using ML approach can be roughly divided into supervised and unsupervised learning methods. The supervised methods make use of a large number of labeled training documents. The unsupervised methods are used when it is difficult to find these labeled training training documents. \par
The lexicon-based approach depends on finding the opinion lexicon which is used to analyze the text. There are two methods in this approach:

\begin{itemize}
\item Dictionary-based
\item Corpus-based
\end{itemize}

In the Dictionary-based approach a small set of opinion words is collected manually with known orientations. Then, this set is grown by searching their synonyms and antonyms. The newly found words are added to the seed list then the next iteration starts. The iterative process stops when no new words are found. The dictionary based approach has a major disadvantage which is the inability to find opinion words with domain and context specific orientation. 

The limitation of the dictionary-based approach is addressed by the corpus-based approach, which depends on syntactic patterns or patterns that occur together along with a seed list of opinion words to find other opinion words in a large corpus. One of these methods is called \textit{sentiment consistency}: it starts with a list of seed opinion adjectives, and used them along with a set of linguistic constraints to identify additional adjective opinion words and their orientations. The constraints being for example \textit{AND, OR, BUT, EITHER-OR,...}; the conjunction \textit{AND} for example says that conjoined adjectives usually have the same orientation. \par

\section{Emotion Detection}

Emotion detection (ED) is the process of identifying human emotions. It is a recent field of research that is closely related to Sentiment Analysis. Indeed, Sentiment Analysis aims to detect positive, neutral or negative feelings from text, whereas Emotion Analysis aims to detect and recognize feelings in natural language texts. Therefore we can look at ED as a finer grained task with respect to SA.\par
Emotion is expressed as joy, sadness, anger, surprise, hate, fear and so on. Since there is not any standard emotion word hierarchy, the focus is on the related research about emotion in cognitive psychology domain. In 2001, W. Gerrod Parrot, wrote a book named ``Emotions In Social Psychology''\cite{Parrott2016}, in which he explained the emotion system and formally classified the human emotions through an emotion hierarchy in six classes at primary level which are \textit{Love, Joy, Anger, Sadness, Fear and Surprise} \cite{edfromtext}.\par

Emotion detection may have useful applications, such as \cite{microsoft}: 
\begin{itemize}
\item Measure citizens happiness;
\item Pervasive computing: this may include suggesting help when anxiety is detected through speech, or to check the tone of an email; 
\item Improving perception of a customer to increase brand reputation and sales. 
\end{itemize}

Some of the biggest challenges in determining emotion are:
\begin{itemize}
\item \textit{Context-dependence of emotions}: people use different regulation strategies in different social contexts. A phrase can have element of \textit{anger} without using the word ``anger'' or any of its synonyms, e.g. \textit{``Shut up!''}
\item \textit{Word-sense disambiguation}: identifying which sense a word (i.e. its meaning) is used in a sentence, when the word has multiple meanings; 
\item \textit{Co-reference resolution}: pronouns and other referring expressions must be connected to the right individuals;
\item Lack of labelled emotion databases. 
\end{itemize}

The main methods used for text based emotion detection are: 
\begin{itemize}
\item \textit{Keyword Spotting}
\item \textit{Lexical Affinity}
\item \textit{Learning-based}
\item \textit{Hybrid}
\end{itemize}

\paragraph{Keyword Spotting}
The keyword pattern matching problem can be described as the problem of finding occurrences of keywords from a given set as substrings in a given string. These words are classified into categories such as disgusted, sad, happy, angry, fearful, surprised, etc. The process of Keyword spotting method is  shown in Fig \ref{fig:keyword_spotting}. 

\begin{figure}[H]
\centering
\includegraphics[width=0.4\textwidth]{./chapters/chapter1/images/keyword_spotting}
\caption{Keyword spotting technique}
\label{fig:keyword_spotting}
\end{figure}

The first step is converting data into tokens, i.e. a sentence into words, then from these tokens emotion words are detected. The second step is analyzing the intensity of emotion words. Sentence, then, is checked whether negation is involved in it or not then finally an emotion class will be assigned.

\paragraph{Lexical Affinity method} 
The Lexical Affinity approach is an extension of keyword spotting technique: apart from picking up emotional keywords it assigns \textit{probabilistic affinity} for a particular emotion to arbitrary words. This technique has the main disadvantage of missing out emotion content that resides deeper than the word level.\\
For example the word 'accident', having been assigned a high probability of indicating a negative emotion, would not contribute correctly to the emotional assessment of phrases like \textit{``I avoided an accident''} or \textit{``I met my girlfriend by accident''}. 

\paragraph{Learning-based methods}
Learning-based methods change the focus from ``determining emotions'' to ``classify the input texts into different emotions''. Indeed, learning-based methods try to detect emotions based on a previously trained classifier, which apply various theories of machine learning such as Support Vector Machines (SVMs).

\paragraph{Hybrid Methods}
Since keyword-based methods and na\"{i}ve learning-based methods could not acquire satisfactory results, some systems use hybrid approach by combining both keyword spotting technique and learning based method, which help to improve accuracy. \\
\\

However all these methods have some major limitations:
\begin{itemize}
\item \textit{Ambiguity in Keyword Definitions}: words can have multiple and vague meanings that can change according to different usages and contexts. Moreover emotion labels could have different emotions in some extreme cases such as ironic or cynical sentences; 
\item \textit{Lack of Linguistic Information}: these methods totally ignore syntax structures and semantics that also have influences on expressed emotions. For example the sentences \textit{``He laughed at me''} or \textit{``I laughed at him''} express two totally different meanings;
\item \textit{Incapability of Recognizing Sentences without Keywords}: sentences without any keyword would imply that they do not contain any emotion at all, which is obviously wrong. 
\end{itemize}

Deciding a way to label emotions is another challenging aspect of ED. There are mainly two possible ways to label data \cite{microsoft}:
\begin{enumerate}
\item The label is one between the set of emotions, e.g. \textit{anger, disgust, sad, happy, surpise, fear, neutral};
\item \textit{Slider approach}: the label is composed of percentages for each emotion, as described in Fig \ref{fig:emotion_labeling_sliders}.
\end{enumerate}

\begin{figure}[H]
\centering
\includegraphics[width=0.8\textwidth]{./chapters/chapter1/images/emotion_labeling_sliders}
\caption{Slider approach\cite{microsoft}}
\label{fig:emotion_labeling_sliders}
\end{figure}

The sliders approach certainly offers more information but it comes with additional computational complications. However, the probabilistic assignment produced in the sliders approach can be turned into distinct labels of single emotions, just like those produced by first labeling approach, but of course, we can not move from the first approach to the second one. 


\chapter{Problem Pre-processing}
\section{The problem}
The goal of the semester project is to unravel the emotion pattern underlying the sequences of songs in a playlist using automatic approaches of Emotion Detection on the lyrics.\par
The problem can be divided into two main parts:
\begin{enumerate}
\item Classify emotions for each song based on the lyrics
\item Analyze emotion patterns in the playlists
\end{enumerate}

\section{Related work}
Emotion detection domain has already attracted many researchers from computer science, psychology, cognitive science and so on. \par
Before building our own emotion detection system we start analyzing some already existent classifier. \par
\paragraph{IBM Watson Natural Language Understanding}\cite{ibm_watson}
Watson is a question answering computer system capable of answering questions posed in natural language, developed by IBM.\par
Natural Language Understanding is a collection of APIs that allows to:
\begin{itemize}
\item Recognize the overall sentiment, in a scale from negative to positive [-1, 1];
\item Detect the emotion percentage between: joy, anger, disgust, sadness, fear;
\item Determine keywords ranked by relevance;
\item Extract entities: people, companies, organizations, cities and other information;
\item Classify content into hierarchical categories;
\item Identify general concepts that may not be directly referenced in the text; 
\end{itemize}
Results obtained analyzing Oasis - Wonderwall are illustrated in Fig.


\paragraph{IBM Watson Tone Analyzer}\cite{ibm_watson_tone}
It uses linguistic analysis to detect joy, fear, sadness, anger, analytical, confident, and tentative tones found in text. It allows to select different sources: tweets, online reviews, email messages, or other text. It uses both:
\begin{itemize}
\item the document level: to get a sense of the overall tone;
\item the sentence level: to identify specific areas where tones are the strongest.
\end{itemize}

\paragraph{QEmotion}\cite{qemotion}
Qemotion detects the main emotion of the speech and define the corresponding emotion in terms of temperature. 
\begin{itemize}
\item From $31^{\circ}$ to $40^{\circ}$ $\to$ Happiness
\item From $21^{\circ}$ to $30^{\circ}$ $\to$ Surprise
\item From $11^{\circ}$ to $20^{\circ}$ $\to $ Calm
\item From $6^{\circ}$ to $10^{\circ}$ $\to $ Fear
\item From $-5^{\circ}$ to $5^{\circ}$ $\to $ Sadness
\item From $-14^{\circ}$ to $-6^{\circ}$ $\to $ Anger
\item From $-20^{\circ}$ to $-15^{\circ}$ $\to $ Disgust
\end{itemize}


\section{NLP libraries}
In order to select the best Natural Language Processing library for our purpose we also analyzed pros and cons of the main Natural Language Processing libraries, i.e. NLTK, TextBlob, Standord's CoreNLP and SpaCy. 

\paragraph{NLTK: Natural Language Toolkit}
It is recommended only as an education and research tool. \\
Pros:
\begin{itemize}
\item its modularized structure makes it excellent for learning and exploring NLP concepts; 
\item over 50 corpora and lexicons, 9 stemmers, and a dozens of algorithms to choose from (this can also be considered as a con).
\end{itemize}
Cons: 
\begin{itemize}
\item Heavy library with a steep learning curve;
\item Slow and not production-ready. 
\end{itemize}

\paragraph{TextBlob}
Built on top on NLTK.\\
Pros: 
\begin{itemize}
\item More intuitive; 
\item Gentle learning curve. 
\end{itemize} 

\paragraph{Stanford's CoreNLP}
Java library with Python wrappers. \\
Pros:
\begin{itemize}
\item fast;
\item support for several major languages. 
\end{itemize}

\paragraph{SpaCy}
It is a new NLP library designed to be fast, streamlined and production-ready.\\
Pros:
\begin{itemize}
\item minimal number of options;
\item its philosophy is to only present the best algorithm for each purpose. 
\end{itemize}
Cons:
\begin{itemize}
\item it is new, so its support community is not as large as other libraries, but it is growing very fast.
\end{itemize}





\section{Word embedding techniques}
Word embeddings are a set of feature learning techniques mapping words or phrases from the vocabulary to vectors or real numbers. \par
These techniques map sparse word vectors into continuous space based on the surrounding context. For example if \textit{``salt''} and \textit{``seasoning''} appear within the same context, the model will indicate that \textit{``salt''} is conceptually closer to \textit{``seasoning''}, than another word, say \textit{``chair''}.\par
There are two main embedding libraries: Word2vec and FastText. While Word2vec treats each word in corpus like an atomic entity generating a vector for each word, FastText treats each word as composed of character ngrams, so the vector for a word is made of the sum of this character n grams. \par
For example,  the word vector \textit{``apple''} is a sum of the vectors of the n-grams ``ap'', ``app'', ``appl'', ``apple'', ``ppl', ``pple'', ``ple'', ``le'' assuming 3 and 6 as minimum and maximum ngrams size.\\
The difference between Word2vec and FastText manifests as follows:
\begin{enumerate}
\item \textit{Rare words}: even if words are rare, their character n-grams are still shared with other words - hence the embeddings with FastText can still be good;
\item \textit{Out of vocabulary words}: FastText can construct the vector for a word from its character n-grams even if word does not appear in training corpus;
\item \textit{Hyperparameters choice}: FastText requires to  choose the minimum and maximum n-grams sizes, and this directly impacts the computation time and the memory requirements. 
\end{enumerate}




\section{Public datasets}
A big challenge in emotion detection is the lack of a labelled emotion database to enable active innovation. Currently, few publicly accessible databases are available.\\\textit{MoodyLyrics}\cite{moodylyrics} contains around 2500 songs manually annotated through Amazon Mechanical Turk with 4 different emotion, i.e., happy, sad, angry and relaxed.\\
\textit{EmoInt}\cite{emoint} contains manually annotated tweets classified according to the intensities of anger, fear, joy and sadness. \textit{EmoBank}\cite{emobank} instead contains 10.000 sentences, each of which has been annotated according to both the emotion expressed by the writer and the emotion perceived by the reader. 




\section{Feature Engineering}
Once we were able to collect data to be used as golden standard for our classification process, we focused our attention on feature engineering. Specifically we tried to extract stylometric, structural, orientation and vocabulary based features\cite{features}. Apart from this we also generated a word embedding vector of the words contained in each song's lyric by using SpaCy's\cite{spacy} pre-trained language model based on word2vec\cite{word2vec}.

Here is a comprehensive list of the features we extracted from our dataset, followed by a brief description:

\begin{description}
\item \textbf{Title\_vector}: word embedding vector of the song's title
\item \textbf{Lyric\_vector}: word embedding vector of the lyric content
\item \textbf{\%Rhymes}: defined as the percentage of the number of rhymes over the number of total lines. A rhyme is defined as a rhyme between two following lines
\item \textbf{Line\_count}: number of lines in the lyric
\item \textbf{Word\_count}: number of words in the lyric
\item \textbf{\%Past\_tense\_verbs}: defined as the the percentage of the number of past tense verbs over the total number of verbs
\item \textbf{\%Present\_tense\_verbs}: defined as the the percentage of the number of present tense verbs over the total number of verbs
\item \textbf{\%Future\_tense\_verbs}: defined as the the percentage of the number of future tense verbs over the total number of verbs, where future is just will + base form
\item \textbf{\%ADJ}: percentage of adjectives over the total number of words
\item \textbf{\%ADP}: percentage of adpositions (e.g. in, to, during) over the total number of words
\item \textbf{\%ADV}: percentage of adverbs (e.g. very, tomorrow, down, where, there) over the total number of words
\item \textbf{\%AUX}: percentage of auxiliaries (e.g. is, has (done), will (do), should (do)) over the total number of words
\item \textbf{\%INTJ}: percentage of interjections (e.g. psst, ouch, bravo, hello) over the total number of words
\item \textbf{\%NOUN}: percentage of nouns over the total number of words
\item \textbf{\%NUM}: percentage of numerals over the total number of words
\item \textbf{\%PRON}: percentage of pronouns (e.g. I, you, he, she, myself, themselves, somebody,...) over the total number of words
\item \textbf{\%PROPN}: percentage of proper nouns (e.g. Mary, John) over the total number of words
\item \textbf{\%PUNCT}: percentage of puntuctuation (e.g. ., (, ), ?) over the total number of words
\item \textbf{\%VERB}: percentage of verbs over the total number of words
\item \textbf{Selfish\_degree}: percentage of 'I' pronouns over the total number of pronouns
\item \textbf{\%Echoism}: percentage of echoism over the total number of words, where an echoism is either a sequence of two subsequent repeated words or the repetition of a vowel in a word
\item \textbf{\%Duplicate\_Lines}: number of lines duplicated across the lyric text
\item \textbf{isTitleInLyric}: boolean, true if the title string is also a substring of the lyric
\item \textbf{Sentiment}: sentiment between -1 and 1
\item \textbf{Subjectivity\_degree}: degree of subjectivity of the text
\end{description}

Since the word embedding vectors we generated had length 300, at the end we were able to obtain 623 distinct numerical features for each of the songs in our dataset.

\subsection{Feature Selection}

Having to deal with 623 different features for discriminating songs among 4 classes is probably enough and many features may be redundant or may not bring any useful information to our goal. Indeed, after running many experiments, we tried to keep our models as simple as possible by trying to select the fewer number of features possible.

In the end, we obtained the best results just by using the following features: \textit{Lyric\_vector}, \textit{\%Echoisms}, \textit{\%Duplicate\_Lines}, \textit{isTitleInLyrc}, \textit{\%Past\_tense\_verbs}, \textit{\%Present\_tense\_verbs}, \textit{\%Future\_tense\_verbs}, \textit{\%ADJ}, \textit{\%PUNCT}, \textit{Sentiment} and \textit{Subjectivity\_degree}. This process of feature selection left us with just 310 distinct features per song.













\chapter{Basic Modeling Approaches}

%%%%%%%%%%%%%%%%%%%%%%%%%%%%%%%%%%%%%%%%%%%%%%%%%
%%%%%%%%%%%%%%%%%%%%%%%%%%%%%%%%%%%%%%%%%%%%%%%%%
%%%%%%%%%%%%%%%%%%%%%%%%%%%%%%%%%%%%%%%%%%%%%%%%%
%%%%%%%%%%%%%%%%%%%%%%%%%%%%%%%%%%%%%%%%%%%%%%%%%
%%%%%%%%%%%%%%%%%%%%%%%%%%%%%%%%%%%%%%%%%%%%%%%%%
\section{Initial Classification}

%%%%%%%%%%%%%%%%%%%%%%%%%%%%%%%%%%%%%%%%%%%%%%%%%
%%%%%%%%%%%%%%%%%%%%%%%%%%%%%%%%%%%%%%%%%%%%%%%%%
%%%%%%%%%%%%%%%%%%%%%%%%%%%%%%%%%%%%%%%%%%%%%%%%%
%%%%%%%%%%%%%%%%%%%%%%%%%%%%%%%%%%%%%%%%%%%%%%%%%
%%%%%%%%%%%%%%%%%%%%%%%%%%%%%%%%%%%%%%%%%%%%%%%%%
\section{POS Tagger Validity Check} 

Before digging into more complex types of analysis, we took the time to check if
the tools we were using could be really good for our purpose. 

Specifically, we had some doubts on the Part-Of-Speech (POS) tagger. Indeed, those kind of systems are generarly 
trained on texts coming from sources whose type of language is very different from those
we would expect to find in lyrics. In fact, the SpaCy's POS tagger implemented in the language
model we are using is trained on OntoNotes 5\cite{ontonotes5} and on Common Crawl\cite{common-crawl},
which are both made of pieces of text taken from news, conversational telephone speech, weblogs, 
usenet newsgroups, broadcast and talk shows. Obviously, this type of natural language texts are 
much different from a lyric and we just wanted to make sure that we were using a appropriate tool.

Before going into the details of what we did, we must state that SpaCy's POS tagger provides two tags per words, which
will both be considered in our analysis. Those two type of tags are:

\begin{itemize}
\item \textbf{POS}: coarse-grained part-of-speech e.g. VERB
\item \textbf{TAG}: fine-grained part-of-speech e.g. PAST\_TENSE
\end{itemize}

In order to obtain reliable insights of the functionalities of our POS tagger we considered three songs:
one with a common language and very few slang words, one filled with slangs and another one with some
vulgar words.

The first song we considered was "Polly" from Nirvana.

As a first approach we tried to tag the words considering one line of the lyric at the time. Here is an example of
what we obtained:

\begin{lstlisting}
Polly wants a cracker
PROPN VERB DET NOUN 

Polly = PROPN NNP -> noun, proper singular
wants = VERB VBZ -> verb, 3rd person singular present
a = DET DT -> determiner
cracker = NOUN NN -> noun, singular or mass
\end{lstlisting}

As a first attempt, our POS tagger completelly succedeed in recognizing the phrase the 
exact way we were expecting. In fact, the absence of weird words e.g. slangs made the
task much easier.

Because of the almost complete absence of punctuation marks in our lyric, we expected
the POS tagger to fail while analyzing the entire song as whole. Instead, we were quite
surprised to see that SpaCy's POS tagger does one desirable thing for our goal:
it treats each line as a standalone sentence, evern though they are not specifically separed by
a stopping mark. Therefore, the same positive behaviour we observed on the first line
was totally replicated on the other lines, giving us the exact tagging we were
expecting by visually inspecting our song's lyrics.

We omitted the entire tagging process output for brevity reasons.

Our first experiment served to the purpose of arriving to a conclusion: SpaCy's POS tagger
is a good tool for lyrics. However it did not solve our doubts about its ability of recognizing
"weirder" words such as slang words or abbreviations. For this reason, we moved on 
analysing the lyrics of "Kiss You Back" from Underground Kiss.

The first interesting thing we noticed was that the POS tagger properly recognized abbreviations such as "'ll'".
Moreover, another important feature we noticed was clearly visible while tagging this line: "Yeah, we chocolate cross-over".
Indeed, here the word "chocolate" is used as a verb (even though chocolate is clearly not defined as a verb in 
the dictionary) and the POS tagger was able to recognize this exception. 
This is quite important because, in songs, those situations happen very often.

Other additional things we noticed while analyzing this song came our of the tagging output of the following line:

\begin{lstlisting}
Jus't havin' fun with it, man, know what I'm sayin'?
NOUN VERB NOUN ADP PRON PUNCT INTJ PUNCT VERB NOUN PRON VERB VERB PUNCT PUNCT 

Jus't = NOUN NNS -> noun, plural
havin' = VERB VBG -> verb, gerund or present participle
fun = NOUN NN -> noun, singular or mass
with = ADP IN -> conjunction, subordinating or preposition
it = PRON PRP -> pronoun, personal
, = PUNCT , -> punctuation mark, comma
man = INTJ UH -> interjection
, = PUNCT , -> punctuation mark, comma
know = VERB VB -> verb, base form
what = NOUN WP -> wh-pronoun, personal
I = PRON PRP -> pronoun, personal
'm = VERB VBP -> verb, non-3rd person singular present
sayin = VERB VBG -> verb, gerund or present participle
' = PUNCT '' -> closing quotation mark
? = PUNCT . -> punctuation mark, sentence closer
\end{lstlisting}

One very interesting result we can notice comes from the following two lines:
\begin{lstlisting}
havin' = VERB VBG -> verb, gerund or present participle and
ayin = VERB VBG -> verb, gerund or present participle
\end{lstlisting}

In fact it looks like our POS tagger is able to recognize verbs in their correct tense
even if they are abbreviated in an unconventional way.

One thing which really impressed us, was the word "man" being recognized to be an interjection from time to time. 
"An interjection is a part of speech that shows the emotion or feeling of the author. These words or phrases can 
stand alone or be placed before or after a sentence. Many times an interjection is followed 
by a punctuation mark, often an exclamation point"\footnote{http://examples.yourdictionary.com/examples-of-interjections.html}. 
This description perfectly fits with the usage of the word "man" in their contextes when it was recognized to be an interjection. 

Those kind of things are not trivial to detect and this ability of our POS tagger convinced us even more
of its impressive skills.

The last thing we were left to analyze at this point was a vulgar song. For this purpose we considered 
"The Ballad Of Chasey Lain", from Bloodhound Gang. 

In this case we have no special remarks to report. We can just say that everything was tagged and
recognized in the exact way we expected.

We did not report entire lyrics nor the full POS tagger output for the interest of brevity. However, those
analysis are available on the public GitHub repository for this project\footnote{https://github.com/sgiammy/emotion-patterns-in-music-playlists/blob/master/Notebook/3\_POS\_tagger\_verification.ipynb}

%%%%%%%%%%%%%%%%%%%%%%%%%%%%%%%%%%%%%%%%%%%%%%%%%
%%%%%%%%%%%%%%%%%%%%%%%%%%%%%%%%%%%%%%%%%%%%%%%%%
%%%%%%%%%%%%%%%%%%%%%%%%%%%%%%%%%%%%%%%%%%%%%%%%%
%%%%%%%%%%%%%%%%%%%%%%%%%%%%%%%%%%%%%%%%%%%%%%%%%
%%%%%%%%%%%%%%%%%%%%%%%%%%%%%%%%%%%%%%%%%%%%%%%%%


\chapter{Model improvements}


\section{Feature Engineering}
In order to improve our model performances, we focused our attention on feature engineering. Specifically we tried to extract stylometric, structural, orientation and vocabulary based features\cite{features}. Apart from this we also generated a word embedding vector of the words contained in each song's lyric by using SpaCy's\cite{spacy} pre-trained language model based on word2vec\cite{word2vec}.\par

Here is a comprehensive list of the features we extracted from our dataset, followed by a brief description:

\begin{description}
\item \textbf{Title\_vector}: word embedding vector of the song's title
\item \textbf{Lyric\_vector}: word embedding vector of the lyric content
\item \textbf{\%Rhymes}: defined as the percentage of the number of rhymes over the number of total lines. A rhyme is defined as a rhyme between two following lines
\item \textbf{Line\_count}: number of lines in the lyric
\item \textbf{Word\_count}: number of words in the lyric
\item \textbf{\%Past\_tense\_verbs}: defined as the the percentage of the number of past tense verbs over the total number of verbs
\item \textbf{\%Present\_tense\_verbs}: defined as the the percentage of the number of present tense verbs over the total number of verbs
\item \textbf{\%Future\_tense\_verbs}: defined as the the percentage of the number of future tense verbs over the total number of verbs, where future is just will + base form
\item \textbf{\%ADJ}: percentage of adjectives over the total number of words
\item \textbf{\%ADP}: percentage of adpositions (e.g. in, to, during) over the total number of words
\item \textbf{\%ADV}: percentage of adverbs (e.g. very, tomorrow, down, where, there) over the total number of words
\item \textbf{\%AUX}: percentage of auxiliaries (e.g. is, has (done), will (do), should (do)) over the total number of words
\item \textbf{\%INTJ}: percentage of interjections (e.g. psst, ouch, bravo, hello) over the total number of words
\item \textbf{\%NOUN}: percentage of nouns over the total number of words
\item \textbf{\%NUM}: percentage of numerals over the total number of words
\item \textbf{\%PRON}: percentage of pronouns (e.g. I, you, he, she, myself, themselves, somebody,...) over the total number of words
\item \textbf{\%PROPN}: percentage of proper nouns (e.g. Mary, John) over the total number of words
\item \textbf{\%PUNCT}: percentage of puntuctuation (e.g. ., (, ), ?) over the total number of words
\item \textbf{\%VERB}: percentage of verbs over the total number of words
\item \textbf{Selfish\_degree}: percentage of 'I' pronouns over the total number of pronouns
\item \textbf{\%Echoism}: percentage of echoism over the total number of words, where an echoism is either a sequence of two subsequent repeated words or the repetition of a vowel in a word
\item \textbf{\%Duplicate\_Lines}: number of lines duplicated across the lyric text
\item \textbf{isTitleInLyric}: boolean, true if the title string is also a substring of the lyric
\item \textbf{Sentiment}: sentiment between -1 and 1
\item \textbf{Subjectivity\_degree}: degree of subjectivity of the text
\end{description}

Since the word embedding vectors we generated had length 300, at the end we were able to obtain 623 distinct numerical features for each of the songs in our dataset.

\subsection{Feature Selection}

Having to deal with 623 different features for discriminating songs among 4 classes is probably enough and many features may be redundant or may not bring any useful information to our goal. Indeed, after running many experiments, we tried to keep our models as simple as possible by trying to select the fewer number of features possible.

In the end, we obtained the best results just by using the following features: \textit{Lyric\_vector}, \textit{\%Echoisms}, \textit{\%Duplicate\_Lines}, \textit{isTitleInLyrc}, \textit{\%Past\_tense\_verbs}, \textit{\%Present\_tense\_verbs}, \textit{\%Future\_tense\_verbs}, \textit{\%ADJ}, \textit{\%PUNCT}, \textit{Sentiment} and \textit{Subjectivity\_degree}. This process of feature selection left us with just 310 distinct features per song.

\section{Beyond the lyrics dataset: EmoInt}

One major limitation we had to face during our work on this project was the shortage in terms of data.
Indeed, most of the already labeled datasets which can be found online have been created for type of texts
which were much different from lyrics i.e. news items, blog posts, Facebook posts, tweets, etc.

In order to overcome this limitation we tought that, if we found a dataset whose items language is close enough
to the common lyrics language, we would have had some performances improvements in our classifiers. This is the
reason why we tried to combine our dataset together with EmoInt\cite{emoint}.

EmoInt provides several tweets annotated according to an emotion (anger, fear, joy, sadness) and to the degree 
at which the emotion is expressed in text. As EmoInt provide intensity levels together with emotion labels, 
we decided to take into account only those tweets for which the intensity was greater that 0.50 (50\%). 

Our original dataset, MoodyLyrics, contains "happy", "sad", "angry" and "relaxed" as labels. 
Therefore, in order to perform a sort of interjection with EmoInt, we used only the tweets corresponding to the anger, 
joy and sadness emotions, discarding those belonging to the fear emotion as we would not have been able to map them into
our original work.

Moreover, it is important to mention that EmoInt was manually annotated using Best-Worst Scaling (BWS)\cite{bws}, 
an annotation scheme proved to obtain very reliable scores. Therefore, we choose EmoInt because it looked
like a realiable choice.

As a single preprocessing technique, we dropped hashtags and remove the tag characters 
(e.g. "Hey @MrTwitter how are you? \#cool" became "Hey MrTwitter how are you?") because 
we had to compare tweets with songs and songs do not have those kind of things. Also, 
this sort of preprocessing should maximize the chances that everything is properly recognized by our POS tagger.

After having preprocessed EmoInt, we combined it to our lyrics based dataset (MoodyLyrics) and tried some different
modeling approaches to see if we could obtain any performance improvements.

After having performed several different trials, we came to the conclusion that the best subset of features
to use for our new dataset (MoodyLyrics + EmoInt) was the following: \textit{Lyric\_Vector}, \textit{Word\_Count}, 
\textit{\%Echoisms}, \textit{Selfish\_degree}, \textit{\%Duplicate\_Lines}, \textit{isTitleInLyric}, 
\textit{\%Present\_tense\_verbs}, \textit{\%Past\_tense\_verbs}, \textit{\%Future\_tense\_verbs},
\textit{\%ADJ}, \textit{\%PUNCT}, \textit{Sentiment} and \textit{Subjectivity\_degree}.

\begin{table}[]
\centering
\label{table:emoint-results}
\begin{tabular}{@{}lll@{}}
\toprule
\textbf{Classifier} & \textbf{Accuracy})   \\ \midrule
k-Nearest Neighbour & 46\%  \\
Support Vector Machine & 48\%  \\
Gradient Boost & 46\%  \\
Neural Network & 82.38\%  \\
Multinomial Na\"{i}ve Bayes Classifier & 49\%  \\ \bottomrule
\end{tabular}
\caption{Accuracy results for different classifiers on MoodyLyrics and EmoInt combined}
\end{table}

Using the data obtained as just explained, we built several classifiers with which we obtained the results
shown in table \ref{table:emoint-results}. Those results were obtained by performing a leave-one-out
validation on the built model.

As it was when we used MoodyLyrics alone, the best results were obtained in using a Neural Network also in this case.
This network had a very simple architecture: one input layer with sigmoid activation function and 120 neurons,
one hidden layer with softmax activation function and 60 neurons and an output layer with 4 neurons and softmax
activation function.

Anyway, it is quite clear that this approach did not lead us to real improvements over our previous cases.
This experiment served to the purpose of understanding that EmoInt is probably too much different from what 
we have classify and it may not improve our predictive abilities at all. Therefore we believed that the best
choice was to keep using a lyrics based dataset as it is MoodyLyrics.



\section{MoodyLyrics duplicates bug}
During the analysis of MoodyLyrics described in the previous chapter we detected the presence of duplicated songs inside the dataset. Moreover, sometimes different emotions were associated with the duplicated songs. Thus, to continue our analysis we eliminated duplicated rows and we chose as emotion label the most frequent emotion between all the duplicates.\par
After reporting the bug to MoodyLyrics owners we have been suggested to use MoodyLyrics4Q, that, according to the creators is a more accurate version of MoodyLyrics.\par
This advice opened us three possibilities: continue using MoodyLyrics, start using MoodyLyrics4Q or create a new dataset as the concatenation of the previous two. We decided to start using all these three models, in order to understand which one, at the end, will give us a better playlists classification. The complete MoodyLyrics emotion classification analysis can be found at \href{https://github.com/sgiammy/emotion-patterns-in-music-playlists/blob/master/Notebook/1_ED_in_songs_lyrics.ipynb}{Notebook 1} while the MoodyLyrics4Q and the emotion detection analysis in the merged datasets can be found at \href{https://github.com/sgiammy/emotion-patterns-in-music-playlists/blob/master/Notebook/2_Advanced_Feature_Engineering.ipynb}{Notebook 2}.


\section{MoodyLyrics4Q}
MoodyLyrics4Q contains 2000 songs and has the same annotation schema as MoodyLyrics. Fig \ref{fig:stats}
shows the emotions distribution comparison between the two MoodyLyrics versions.

\begin{figure}[H]
\centering
\includegraphics[width=1.1\textwidth]{./chapters/chapter5/images/Stats.png}
\caption{Emotions distribution comparison between MoodyLyrics and MoodyLyrics4Q}
\label{fig:stats}
\end{figure}

MoodyLyrics4Q classes are more much balanced, however MoodyLyrics4Q contains only 2000 songs instead of 2509. \par
We studied the qualitative difference between the two version comparing the classification given to the songs contained in both datasets to establish what version, according to us, is more correct. The intersection between the two versions contains 47 songs, and 21 over 47 have been classified differently. We noticed that in 15 over this 21 songs the two datasets confuses \textit{happy} with \textit{relaxed} and \textit{angry} with \textit{sad}. Indeed, only 6 of 21 songs are classified totally differently, however reading the lyrics of each of this song we could not establish which version is the best one.



\section{Results}
In this section we present the result obtained while predicting one of the four emotion labels \textit{relaxed}, \textit{happy}, \textit{sad}, \textit{angry}, using an artificial neural network, a support vector machine, the logistic regression and xgboost. The accuracies have been computed with a 5-fold cross validation. All the implementation details can be found at \href{https://github.com/sgiammy/emotion-patterns-in-music-playlists/blob/master/Notebook/2_Advanced_Feature_Engineering.ipynb}{Notebook 2}.

\begin{table}[H]
\begin{tabular}{ |p{3cm}||p{1.5cm}|p{1.5cm}|p{1.5cm}|p{1.5cm}|  }
 \hline
 \multicolumn{5}{|c|}{5-fold Cross Validation Accuracy} \\
 \hline
 Dataset & ANN & LR &SVM & xgboost\\
 \hline
MoodyLyrics4Q  & 51\%    &55\% &  59\% & 56\%\\
Both together &   67\%  & 68\%   &69\% &63.7\%\\
\hline
\end{tabular}
\caption{Emotion detection accuracies} \label{tab:comparison}
\end{table}

\begin{figure}[H]
  \centering
  \begin{subfigure}[b]{0.49\linewidth}
    \includegraphics[width=\linewidth]{./chapters/chapter5/images/4Q/CM_ANN.png}
    \caption{ML4Q}
  \end{subfigure}
  \begin{subfigure}[b]{0.49\linewidth}
   \includegraphics[width=\linewidth]{./chapters/chapter5/images/join/CM_ANN.png}
    \caption{ML + ML4Q}
  \end{subfigure}
  \caption{Artificial Neural Network - Confusion Matrix}
  \label{fig:ann}
\end{figure}

\begin{figure}[H]
  \centering
  \begin{subfigure}[b]{0.49\linewidth}
    \includegraphics[width=\linewidth]{./chapters/chapter5/images/4Q/CM_LR.png}
    \caption{ML4Q}
  \end{subfigure}
  \begin{subfigure}[b]{0.49\linewidth}
   \includegraphics[width=\linewidth]{./chapters/chapter5/images/join/CM_LR.png}
    \caption{ML + ML4Q}
  \end{subfigure}
  \caption{Logistic Regression - Confusion Matrix}
  \label{fig:lr}
\end{figure}

\begin{figure}[H]
  \centering
  \begin{subfigure}[b]{0.49\linewidth}
    \includegraphics[width=\linewidth]{./chapters/chapter5/images/4Q/CM_SVM.png}
    \caption{ML4Q}
  \end{subfigure}
  \begin{subfigure}[b]{0.49\linewidth}
   \includegraphics[width=\linewidth]{./chapters/chapter5/images/join/CM_SVM.png}
    \caption{ML + ML4Q}
  \end{subfigure}
  \caption{Support Vector Machine - Confusion Matrix}
  \label{fig:svm}
\end{figure}

\begin{figure}[H]
  \centering
  \begin{subfigure}[b]{0.49\linewidth}
    \includegraphics[width=\linewidth]{./chapters/chapter5/images/4Q/CM_XGB.png}
    \caption{ML4Q}
  \end{subfigure}
  \begin{subfigure}[b]{0.49\linewidth}
   \includegraphics[width=\linewidth]{./chapters/chapter5/images/join/CM_XGB.png}
    \caption{ML + ML4Q}
  \end{subfigure}
  \caption{Xgboost - Confusion Matrix}
  \label{fig:xgb}
\end{figure}





\chapter{Conclusions and Future Works}

In this report we detailed all the experiments and trials we did in the context
of our semester project. Our main goal was to firstly build a system capable of
classifying songs lyrics based on their emotion, and then to use the acquired knowledge 
to understand emotional patterns in music playlists.

The first thing we did, was focusing on the word embedding vectors generated for the lyrics
of the already labeled songs we found in the MoodyLyrics dataset, which classifies lyrics according to 
4 possible emotions: angry, sad, relaxed and happy. As we understood that this
approach was very limiting, we started working on some smarter feature engineering approaches,
in which we extracted our own features from the analyzed lyrics.

Under the suggestion of the dataset's creator, we then moved from MoodyLyrics to its updated and more
reliable version: MoodyLyrics4Q. This new dataset became then the gold standard for our experiments.

We tried to combine our lyrics dataset also with data coming from different data sources, e.g. tweets. However
we understood that the best way to proceed in our project was to keep working with just the lyrics
based dataset, as it is obviously the most suitable one for unraveling emotional patters in music playlists.

At the end of the lyrics classification process, our output was, for each song, an emotion vector, composed 
of the intensities at which each sentiment was expressed.

Once we understood which was the best subset of features and the best classifiers, we moved our focus on
classifying playlists. For evaluating our playlist classification performances, we compared our outputs
to those of the silver standard playlist dataset we built.

We used two different approaches for playlist classification: in the first one we
just computed the emotion classification vector for each song in the playlist and then we averaged them;
in the second approach, before doing the average, we exclude from the emotion vectors those values which 
were considered to be too different from the others (outliers).

Before having a look at the emotional patterns in the playlists, we tried to ease out our computational
load by trying to see if it was possible to properly classify a playlist based on just a subset of its songs.
However, we found out that we need all the songs in each playlist if we want to achieve the best results possible.

To conclude, we explored the emotional patterns inside each playlist, meaning that we saw how the emotion
classification vector evolved inside each playlist, transitioning from one emotion to another.

In conclusion, we proved that the best results we could get were those obtained by classifying lyrics using the Artificial Neural Network we described in the previous chapters. Indeed, despite the fact that it could be slightly outperformed by a linear regression classifier when considering just lyrics, it was the algorithm that provided us with the best accuracy results on playlists (which was our ultimate goal). In fact, we were able to obtain a 55.97\% classification accuracy on lyrics and a 66\% classification accuracy on playlists.


%%%%%%%%%%%%%%%%%%%%%%%%%%%%%%%%%%%%%%%%%%%%%%%%%%%%
%%%%%%%%%%%%%%%%%%%%%%%%%%%%%%%%%%%%%%%%%%%%%%%%%%%%
%%%%%%%%%%%%%%%%%%%%%%%%%%%%%%%%%%%%%%%%%%%%%%%%%%%%

Despite the interesting results we obtained, there is still much room
for improvements.

Indeed, one thing we did not explore in much depth is the usage of Recurrent 
Neural Networks (RNN) and Convolutional Neural Networks (CNN), employed by
many modern systems built for emotion classification tasks.

We made several small experiments using some complex CNN and RNN based networks
we found in the literature\cite{text-emotion-classification}. However, those models produced very poor performances,
which is the reason why we did not even mention them in the current report.

Anyway, those small experiments served to the purpose of letting us understand 
which direction to take in the future. Indeed, neural networks components, such as
Long Short Term Memories (LSTMs), seems to be perfectly suited for those kind of problems
and, if properly tuned, may help us in reaching better performances.

One current limitation we saw when running the mentioned experiments with deep neural
networks is the shortage of training data. Indeed, effective training of 
neural networks requires a huge amount of data. In the low-data regime, parameters 
are underdetermined, and learnt networks generalise poorly. 

The simplest way to solve this problem, would be to manually label more and more song lyrics, eventually using
automated mechanisms e.g. Amazon Mechanical Turk\cite{amazon-turk}. However this solution is not practical
and expensive (especially if we use Amazon Mechanical Turk).

Certainly, a better approach would be to use some automated data generation techniques.
Indeed, we may think of using some of the Data Augmentation mechanisms currently employed in deep learning. 
We found several interesting techniques\cite{DBLP:journals/corr/abs-1711-00648} which could be employed in our domain
and could help alleviate issue by using Generative Adversarial Neural Networks (GANN), which
are able to generate new data from the already existing one.


\bibliography{main}{}
\bibliographystyle{plain}
 

\end{document}
              
